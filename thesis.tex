% Related work:
% Object representation of scope during translation (coined "scope graph"):
%   http://www.lirmm.fr/~ducour/Doc-objets/ECOOP/papers/0276/02760071.pdf
% MetaOcaml & related work by Walid Taha
% Interesting sugary languages: Dylan, Julia, Nim
% From Joe, "See S8. And also the whole paper.":
%   https://www.microsoft.com/en-us/research/wp-content/uploads/2016/11/compiling-without-continuations.pdf
% PHOAS?
%   Chipala ICFP'08 https://archive.alvb.in/msc/10_infomepcs1/literature/PHOAS_Chlipala.pdf
% Extensible Grammars for Language Specialization (must CITE)
%   Cardelli, Matthes, Abadi - DigEquipRC

\RequirePackage{silence} % :-\
    \WarningFilter{scrbook}{Usage of package `titlesec'}
    \WarningFilter{titlesec}{Non standard sectioning command detected}
\documentclass[
  11pt,
  paper=letter,
  footinclude=true,
  headinclude=true,
  american
]{scrbook}

\usepackage[T1]{fontenc}
\usepackage[
  linedheaders=true,
  parts=true
]{classicthesis/classicthesis} % ,manychapters

\usepackage{amsthm}
\usepackage{amsmath}
\usepackage{amssymb}
\usepackage{xargs}
\usepackage{xspace}
\usepackage{semantic}
\usepackage{cleveref}
\usepackage{color}
\usepackage{alltt}
\usepackage{listings}
\usepackage{multicol}
\usepackage{bussproofs}



% Theorems
\newtheorem{definition}{Definition}
\newtheorem{theorem}{Theorem}
\newtheorem{lemma}{Lemma}
\newtheorem{assumption}{Assumption}

\lstset{basicstyle=\ttfamily,breaklines=true}
\lstset{framextopmargin=50pt,frame=bottomline}

% Formatting
\newcommand{\desc}[1]{\noindent\textit{#1:} }
\newenvironment{jtable}
{\begin{center}\begin{tabular}{l l l @{\quad}l}}
{\end{tabular}\end{center}}
               

% Common math notation
\newcommand{\ddd}{\;\dots\;}
\newcommand{\dd}{\,...\,}

% Common math notation
\newcommand{\Exists}[1]{\exists{#1}.\,\,}
\newcommand{\ExistsUnique}[1]{\exists!{#1}.\,\,}
\newcommand{\NotExists}[1]{\not\exists{#1}.\,\,}
\newcommand{\Forall}[1]{\forall{#1}.\,\,}
\newcommand{\NotForall}[1]{\not\forall{#1}.\,\,}
\newcommand{\SetSuchThat}[2]{\{#1 \;|\; #2\}}
\newcommand{\domain}[1]{\textit{domain}(#1)}
\newcommand{\DisjUnion}{\,\dot\cup\,}
\newcommand{\To}{\Rightarrow}

% Common code notation
\newcommand{\code}[1]{\texttt{#1}}
\newenvironment{codes}
  {\begin{alltt}\leftskip=1.5em} % \small
  {\end{alltt}}

% Math names
\newcommand{\op}[2]{\textit{#1}(#2)}
\newcommand{\opName}[1]{\textit{#1}}
\newcommand{\name}[1]{\textit{#1}}
\newcommand{\constName}[1]{\texttt{#1}}

% Expressions & Grammars
\newcommand{\lit}[1]{\textbf{#1}}
\newcommand{\expr}[2]{(#1\,#2)}
\newcommand{\var}[1]{\textrm{\textsc{#1}}}

% Grammars
\newcommand{\exprs}[3]{(#1\,#2\,#3^{*})}
\newcommand{\production}[2]{#1 \leftarrow #2}
\newcommand{\saysG}[3]{#1 \vdash #2\,:\,#3}

\definecolor{goodRed}{rgb} {0.70, 0.37, 0.41} % LAB (50, 35, 10)
\definecolor{goodBlue}{rgb}{0.04, 0.50, 0.70} % LAB (50, -10, -35)
\hypersetup{
  colorlinks=false,
  %citecolor=goodBlue,
  %linkcolor=goodRed
  citebordercolor=goodBlue,
  linkbordercolor=goodRed
  }


\begin{document}

\author{Justin Pombrio}
\title{Resugaring: Restoring the Abstractions that Syntactic Sugar is Supposed to Provide}
\maketitle

\part{Syntactic Sugar}
\include{chapters/chapter01}
\chapter{Desugaring in the Wild}

.[TODO: State version of each system discussed]

\section{C Preprocessor} \label{sec:cpre}


\desc{Evaluation Strategy} IO

\desc{Authorship} User-defined

\desc{Representation} Token stream

\desc{Safety} [FILL]

% https://gcc.gnu.org/onlinedocs/cpp/
% Not Turing complete: https://gcc.gnu.org/onlinedocs/cpp/Self-Referential-Macros.html
% Evaluation strategy: https://gcc.gnu.org/onlinedocs/cpp/Macro-Arguments.html
\desc{Discussion}
The C Preprocessor (hereafter \textsc{cpp}) [CITE] is a \emph{text
  preprocessor}: a source-to-source transformation that operates at
the level of text. (More precisely, it operates on a token stream, in
which the tokens are approximately those of the C language). It is
usually run before compilation for C or C++ programs, but it is not
very language specific, and can be used for other purposes as well.
\textsc{Cpp} is not Turing complete, by a simple mechanism: if a macro
invokes itself (directly or indirectly), the recursive invocation
will not be expanded.

%https://stackoverflow.com/questions/14041453/why-are-preprocessor-macros-evil-and-what-are-the-alternatives
A number of issues arise from the fact that \textsc{cpp} operates on tokens, and
is thus unaware of the higher-level syntax of C [CITE].
As an example, consider this innocent looking
\textsc{cpp} desugaring rule that defines an alias for subtraction:
\begin{codes}
  #define SUB(a, b) a - b
\end{codes}
This rule is completely broken. Suppose it is used as follows:
\begin{codes}
  SUB(0, 2 - 1))
\end{codes}
This will expand to \code{0 - 2 - 1} and evaluate to \code{-3}.
We can revise the rule to fix this:
\begin{codes}
  #define SUB(a, b) (a) - (b)
\end{codes}
This will fix the last example, but it is still broken. Consider:
\begin{codes}
  SUB(5, 3) * 2
\end{codes}
This will expand to \code{5 - 3 * 2} and evaluate to \code{-1}.
The rule can be fully fixed by another set of parentheses:
\begin{codes}
  #define SUB(a, b) ((a) - (b))
\end{codes}
In general, both the inside boundary of a rule (the arguments \code{a}
and \code{b}), and the outside boundary (the whole \textsc{rhs}) need
to be protected to ensure that the expansion is parsed correctly. If
the sugar is used in expression position, as in the \code{SUB}
example, this can be done with parentheses. In other positions,
different tricks must be used: e.g., a rule meant to be used in
statement position can be wrapped in \code{do \{...\} while(0)}.
Software developers should not need to know this.

There are other issues that arise with text-based transformations as
well, such as variable capture. Furthermore, all of these issues are
inherent to text-based transformations, and essentially cannot be
fixed from within the paradigm. \emph{Overall, code transformations
  should never operate at the level of text.}

\section{C++ Templates} \label{sec:cpp}

\desc{Representation} Concrete Syntax

\desc{Authorship} User-defined

\desc{Evaluation Strategy} IO

\desc{Safety} [FILL]

% http://www.open-std.org/jtc1/sc22/wg21/docs/papers/2017/n4659.pdf
\desc{Discussion} C++ templates [CITE] are not general-purpose sugars,
because they cannot take code as an argument.  Instead, they are used
primarily to instantiate polymorphic code by replacing type parameters
with concrete types.  Let's use the following template declaration,
taken from [CITE: pg344], as a running example. It declares a function
to compute the area of a circle, that can be instantiated with
different possible (presumably numeric) types \code{T}:
\begin{codes}
template<class T>
T circular_area(T r) \{
  return pi<T> * r * r;
\}
\end{codes}

Besides function definitions, several other kinds of declarations can
be templated, including methods, classes, structs, and type aliases.
The behavior of each is similar. A template may be invoked by passing
arguments in angle brackets. An invoked template acts like the
kind of thing the template declared, and can be used in the same
positions. Thus, e.g., a \code{struct} template should be invoked in type
position; and our running function template example should be invoked
in expression position to make a function, which can then be called:
\begin{codes}
  float area = circular_area<float>(1);
\end{codes}
When a template is invoked like this, a copy of the template
definition is made, with the template parameters replaced with the
concrete arguments.\marginpar{
  If a template is invoked multiple times with the same parameters,
  only one copy of the code will be made, however.
}
In our example, this produces the code:
\begin{codes}
float circular_area(float r) \{
  return pi<float> * r * r;
\}
\end{codes}

So far we have only described type parameters, but templates can also
take other kinds of parameters, including primitive values (such as
numbers) and other templates. The ability to manipulate numbers and
invoke other templates at compile time make C++ templates powerful
and, unsurprisingly, Turing complete. However, templates \emph{cannot}
be parameterized over code, and thus are not general-purpose sugars.
For example, most of the examples in this thesis cannot be written as
C++ templates.

Template expansion uses IO evaluation order. This is important because
it is possible
to define both a generic template, that applies most of the time, and
a specialized template, that applies if a parameter has a particular
value. For example, this could be used to make a \code{HashMap} use a different
implementation if its keys are \code{int}s. Thus it is important that
a template see the concrete type (e.g. \code{int}) that is passed to
it, even if this type is the result of another template expansion.


\section{Rust Macros} \label{sec:rust}

\desc{Representation} Concrete Syntax

\desc{Authorship} User-defined

\desc{Evaluation Strategy} OI

\desc{Safety} [FILL]

%https://doc.rust-lang.org/1.2.0/book/macros.html
\desc{Discussion}


\section{Haskell Templates} \label{sec:haskell}

\desc{Representation} Concrete or Abstract Syntax

\desc{Authorship} User-defined

\desc{Evaluation Strategy} IO

\desc{Safety} AST safe. Scope unsafe. Type unsafe.

%https://hackage.haskell.org/package/template-haskell-2.10.0.0/docs/Language-Haskell-TH-Syntax.html#t:Lit
%https://stackoverflow.com/questions/10857030/whats-so-bad-about-template-haskell
\desc{Discussion}
NOTES:
\begin{itemize}
  \item Must be defined in separate file
\end{itemize}

\chapter{Resugaring for Pyret}

\section{Example}

\subsection{Define-struct}

\paragraph{Core AST}
\begin{Codes}
Stmts:
| [\{splicing-begin stmts:Stmts\} @rest:Stmts]
   binding stmts in rest
   providing stmts, rest

| [\{let x:Var v:Expr\} @rest:Stmts]
   binding x in rest

| [\{fun f:Var args:Args body:Expr\} @rest:Stmts]
   binding args in body, rest in body
   providing f, rest

ALTERNATIVELY:

Stmts:
| \{splicing-begin stmts:Stmts rest:Stmts\}
   binding stmts in rest
   providing stmts, rest

| \{let x:Var v:Expr rest:Stmts\}
   binding x in rest

| \{fun f:Var args:Args body:Expr rest:Stmts\}
   binding args in body, rest in body
   providing f, rest

| \{end\}

Params:
| \{param x:Var rest:Params\}
| \{end\}
\end{Codes}

\paragraph{Auxiliary AST}
\begin{Codes}
IStructFields:
| [field:IStructField ...fields:IStructFields]
  providing field, fields

IStructField:
| \{i-struct-field field:Str get:Var set:Var\}
  providing get, set
\end{Codes}

\paragraph{Surface AST}
\begin{Codes}
SurfStmts:
  .....
| [(define-struct name:Var fields:StructFields) @rest:SurfStmts]
  binding name in rest, fields in rest
  providing name, fields, rest

StructFields:
| [field:StructField ...fields:StructFields]
  providing field, fields

StructField:
| (struct-field field:Str get:Var set:Var)
  providing get, set
\end{Codes}

\paragraph{Desugaring Rules}
\begin{Codes}
   [(struct-field field:Str get:Var set:Var) @rest:IStructFields]
=> [\{i-struct-field field get set\} @rest]
  
   [(define-struct name:Var
      [(struct-field field:Str get:Var set:Var) ...]) @rest:SurfStmts]
=> [\{fun name [x ...] \{record [\{record-field field x\} ...]\}\}
    \{splicing-begin [\{fun get [rec] \{record-get rec field\}\} ...]\}
    \{splicing-begin [\{fun set [rec val] \{record-set rec field val\}\} ...]\}
    @rest]
\end{Codes}

\subsection{Pyret For Expressions}

To handle Pyret for-expressions, we need to do two things.
First, when a for-expression binding (e.g. \Code{n from 0}) desugars,
it will simply return its binding (\Code{n}) and its value (\Code{0})
to the for-expression. It can do so with the desugaring rule:
\begin{Codes}
   (s-for-bind l:Loc b:Bind v:Expr)
=> \{for-bind b v\}
\end{Codes}
where \Code{ForBind} is a new type:
\begin{Codes}
  ForBind ::= \{for-bind Bind Expr\}

with list scope:
  [\{for-bind b v\} ...]
  export b
  export ...
\end{Codes}

Then for-expressions can be implemented with the desugaring rule:
\begin{Codes}
   (s-for l:Loc
          iter:Expr
          [\{ForBind bind:Bind value:Expr\} ...]@binds
          ann:Ann
          body:Expr
          blocky:Bool)
=> \{Lambda l (CONCAT "for-body<" (FORMAT l false) ">")
     [] [bind ...] ann "" body None None blocky\}
with scope:
  bind binds in body
\end{Codes}

Notice that \Code{s-for} is pattern matching against the results of
desugaring the \Code{s-for-bind}s. The \Code{(CONCAT ...)} stuff is to
compute at compile time a name for this lambda, which is what Pyret
currently does.

\section{Expressions}

\begin{Table}
core name $C$ &$::=$& \textit{name} & core syntactic construct name \\
surface name $m$ &$::=$& \textit{name} & surface syntactic construct name \\
expression $e$ &$::=$& $\Core{C}{e_1 ... e_n}$ & core syntactic construct \\
  &$|$& $\Surf{m}{e_1 ... e_n}$ & surface syntactic construct \\
  &$|$& $[e_1 ... e_n]$ & list \\
  &$|$& $string$ & string literal \\
  &$|$& $\Refn[i]{x}$ $|$ $\Decl[i]{x}$  & variable \\
value $v$ &$::=$& $e$ & with no sugar invocations \\
\end{Table}



\section{Expansion}

\begin{Table}
ellipsis label $l$ &$::=$& \textit{name} & ellipsis label \\
%shape $\dot{e}$ &$::=$& ...e... & (same cases as $e$) \\
%  &$|$& $\bullet$ & hole \\
pattern $p$ &$::=$& $\PVarA$ & pattern variable \\
  &$|$& $\Core{C}{p_1 ... p_n}$ & syntactic construct \\
  &$|$& $\Surf{m}{p_1 ... p_n}$ & sugar invocation \\
  &$|$& $[ps]$ & list \\
  &$|$& $string$ & string literal \\
  &$|$& $\Refn[i]{x}$ $|$ $\Decl[i]{x}$  & variable \\
seq. pattern $ps$ &$::=$& $\epsilon$ & empty sequence \\
  &$|$& $\Cons{p}{ps}$ & cons \\
  &$|$& $\Rep{p}{l}$ & ellipsis with label $l$ \\
fresh vars $F$ &$::=$& $\{x,...\}$ & fresh variable set \\
type env. $\Gamma$ &$::=$&
$\begin{cases}
  \PVarA:t, ... \\
  i \mapsto [\Gamma], ...
\end{cases}$ \\
substitution $\gamma$ &$::=$&
$\begin{cases}
  \PVarA \mapsto e, ... \\
  l \mapsto [\gamma ... \gamma], ... \\
  x \mapsto x, ...
\end{cases}$
\end{Table}

\begin{Table}
rewrite case $c$ &$::=$&
  $\DsRuleCase{(p_1,\,...,\,p_k)}{\Gamma}{F}{p'}$ \\
desugaring rule $r$ &$::=$&
  $\DsRule{m}{c_1,...,c_n}$ \\
desugaring rules $rs$ &$::=$& $\{r_1, ..., r_n\}$
\end{Table}

\subsection{Matching and Substitution}

\begin{figure}
\fbox{$\SaysMatch{F}{e}{p}{\gamma}$}
\begin{multicols}{2}
  \Inference[m-pvar]{}{
    \SaysMatch{F}{e}{\PVarA}{\{\PVarA \mapsto e\}}
  }

  \Inference[m-capture]{
    x \not\in F
  }{
    \SaysMatch{F}{x}{x}{\{\}}
  }

  \Inference[m-fresh]{
    x \in F
  }{
    \SaysMatch{F}{y}{x}{\{x \mapsto y\}}
  }

  \Inference[m-str]{}{
    \SaysMatch{F}{string}{string}{\{\}}
  }

  \Inference[m-empty]{}{
    \SaysMatch{F}{[\phantom{.}]}{\epsilon}{\{\}}
  }

  \Inference[m-cons]{
    \SaysMatch{F}{e_1}{p}{\gamma_1} \\
    \SaysMatch{F}{[e_2,...,e_n]}{ps}{\gamma_s} \\
    \gamma_1 \DisjUnion \gamma_2 = \gamma
  }{
    \SaysMatch{F}{[e_1 ... e_n]}{p,ps}{\gamma}
  }
\end{multicols}
\vspace{1em}

\Inference[m-con]{
  \SaysMatch{F}{e_1}{p_1}{\gamma_1} \;\cdots\; \SaysMatch{F}{e_n}{p_n}{\gamma_n}
  \gamma_1 \DisjUnion ... \DisjUnion \gamma_n = \gamma
}{
  \SaysMatch{F}{\Core{C}{e_1 ... e_n}}{\Core{C}{p_1 ... p_n}}{\gamma}
}

\Inference[m-star]{
  \SaysMatch{F}{e_1}{p}{\gamma_1} \;\cdots\; \SaysMatch{F}{e_n}{p}{\gamma_n}
}{
  \SaysMatch{F}{[e_1 ... e_n]}{[\Rep{p}{l}]}{\{l \mapsto [\gamma_1 ... \gamma_n]\}}
}

\fbox{$\SaysSubs{F}{\gamma}{p}{e}$}
\begin{multicols}{2}
  \Inference[s-pvar]{
    \PVarA \mapsto e \in \gamma
  }{
    \SaysSubs{F}{\gamma}{\PVarA}{e}
  }

  \Inference[s-str]{}{
    \SaysSubs{F}{\gamma}{string}{string}
  }

  \Inference[s-capture]{
    x \not\in F
  }{
    \SaysSubs{F}{\gamma}{x}{x}
  }

  \Inference[s-fresh]{
    x \in F & x \mapsto y \in \gamma
  }{
    \SaysSubs{F}{\gamma}{x}{y}
  }

  \Inference[s-empty]{}{
    \SaysSubs{F}{\gamma}{[\epsilon]}{[\phantom{.}]}
  }

  \Inference[s-cons]{
    \SaysSubs{F}{\gamma}{p}{e_1} \\
    \SaysSubs{F}{\gamma}{[ps]}{[e_2,...,e_n]}
  }{
    \SaysSubs{F}{\gamma}{[p,ps]}{[e_1 e_2 ... e_n]}
  }
\end{multicols}
\vspace{1em}

\Inference[s-star]{
  l \mapsto [\gamma_1 ... \gamma_n] \in \gamma \\
  \SaysSubs{F}{\gamma_1}{p}{e_1} \;\cdots\; \SaysSubs{F}{\gamma_n}{p}{e_n}
}{
  \SaysSubs{F}{\gamma}{[\Rep{p}{l}]}{[e_1 ... e_n]}
}

\Inference[s-con]{
  \SaysSubs{F}{\gamma}{p_1}{e_1} \;\cdots\; \SaysSubs{F}{\gamma}{p_n}{e_n}
}{
  \SaysSubs{F}{\gamma}{\Core{C}{p_1 ... p_n}}{\Core{C}{e_1 ... e_n}}
}

\Inference[s-sugar]{
  \SaysSubs{F}{\gamma}{p_1}{e_1} \;\cdots\; \SaysSubs{F}{\gamma}{p_n}{e_n}
}{
  \SaysSubs{F}{\gamma}{\Surf{m}{p_1 ... p_n}}{\Surf{m}{e_1 ... e_n}}
}

\caption{Matching and Substitution}
\end{figure}

\begin{lemma}[matching and substitution]
  Matching and substitution are inverses:
%  $\SaysSubs{F}{\gamma}{p}{e}$, then $\SaysMatch{F}{e}{p}{\gamma}$.
\end{lemma}
\begin{proof}
  Induct on $p$.
  [FILL]
\end{proof}
%However, the reverse is not true. Matching does not undo substitution,
%because substitution in non-deterministic (because it generates fresh
%variables).

\subsection{Expansion}

See \cref{fig:expansion}.
[TODO: Replace step with something that looks like desugaring.]
[TODO: Replace $v$ with something that looks like core terms.]

\begin{figure}
  \Inference[eval-ctx]{
    \SaysStep{L}{e}{e'}
  }{
    \SaysStep{L}{E[e]}{E[e']}
  }
  \Inference[eval-expand]{
    L = G,rs &
    \DsRule{m}{c_1 ... c_n} \in G \IS\\
    \SaysCase{L}{\Surf{m}{e_1 ... e_n}}{e''}{c_i} \IS\\
    \SaysNotCase{L}{\Surf{m}{e_1 ... e_n}}{e'}{c_j} \text{ for any } j<i
  }{
    \SaysStep{L}{\Surf{m}{e_1 ... e_n}}{e'}
  }
  \Inference[eval-case]{
    \SaysMatch{L}{e_i}{p_i}{\gamma_i} \text{ for each $i$} \IS \\
    \gamma' \text{ gives fresh names to the variables in $F$} \\
    \gamma_1 \DisjUnion ... \DisjUnion \gamma_n \DisjUnion \gamma' = \gamma \\
    \SaysSubs{F}{\gamma}{p'}{e'}
  }{
    \SaysCase{L}{\Surf{m}{e_1 ... e_n}}{e'}{(p_1,...,p_n);\Gamma;F \To p'}
  }
  \caption{Expansion}
  \label{fig:expansion}
\end{figure}

[FILL] One expansion rule. Note expansion contexts.



\section{AST Checking} % Or Syntype Checking

\begin{Table}
ast defn. $G$ &$::=$& $A \mapsto \{t_1, ... t_n\}$
  & (with no production $A_1 \mapsto A_2$) \\
syntactic category $A$ &$::=$& \textit{name} \\
syntax type $t$ &$::=$& $A$ & syntactic category \\
  &$|$& $\Core{C}{t_1 ... t_n}$ & syntactic construct \\
  &$|$& $[t]$ & list \\
  &$|$& String & string literal \\
  &$|$& Decl & variable declaration \\
  &$|$& Refn & variable reference \\
language $L$ &$::=$& $G, rs$
\end{Table}

Lemma: If rules grammar check, then e obeys Surf implies ds(e) obeys
Core.

Lemma: Normalizing a grammar does not change its language.

\paragraph{Exhaustion Checking}
We perform exhaustion checking to make sure that sugars cover all
possible cases of their arguments, but do not give the algorithm here.
It works by looking at \emph{shapes}: a shape is a pattern that
contains types in place of pattern variables. It is straightforward to
check whether an expression matches a shape, and to convert a pattern
into a shape. Exhaustion checking uses the fact that the expressions
that do \emph{not} match a shape can be expressed as a union of shapes.
[TODO: prove]

\begin{figure}

\fbox{$\SaysExpr{L}{e}{t}$}

\begin{multicols}{2}
  
  \Inference[e-con]{
    A \mapsto \Core{C}{t_1 ... t_n} \in L \\
    \SaysExpr{L}{e_1}{t_1} \;\cdots\; \SaysExpr{L}{e_n}{t_n}
  }{
    \SaysExpr{L}{\Core{C}{e_1 ... e_n}}{A}
  }

  \Inference[e-refn]{}{
    \SaysExpr{L}{\Refn{x}}{\TRefn}
  }

  \Inference[e-decl]{}{
    \SaysExpr{L}{\Decl{x}}{\TDecl}
  }

  \Inference[e-str]{}{
    \SaysExpr{L}{\textit{string}}{\TString}
  }

  \Inference[e-list]{
    \SaysExpr{L}{e_1}{t} \;\cdots\; \SaysExpr{L}{e_n}{t}
  }{
    \SaysExpr{L}{[e_1 ... e_n]}{[t]}
  }

  \Inference[e-sugar]{
    \SaysRule{L}{m}{t_1,...,t_n}{t} \\
    \SaysExpr{L}{e_1}{t_1} \;\cdots\; \SaysExpr{L}{e_n}{t_n}
  }{
    \SaysExpr{L}{\Surf{m}{e_1 ... e_n}}{t}
  }
\end{multicols}

\fbox{$\SaysPatt{L}{\Gamma}{p}{t}$}

\begin{multicols}{2}

  \Inference[p-pvar]{
    \PVarA : t \in \Gamma
  }{
    \SaysPatt{L}{\Gamma}{\PVarA}{t}
  }

  \Inference[p-refn]{}{
    \SaysPatt{L}{\Gamma}{x}{\TRefn}
  }

  \Inference[p-decl]{}{
    \SaysPatt{L}{\Gamma}{x}{\TDecl}
  }

  \Inference[p-str]{}{
    \SaysPatt{L}{\Gamma}{\textit{string}}{\TString}
  }

  \Inference[p-con]{
    A \mapsto \Core{C}{t_1 ... t_n} \in L \\
    \SaysPatt{L}{\Gamma}{p_1}{t_1} \;\cdots\; \SaysPatt{L}{\Gamma}{p_n}{t_n}
  }{
    \SaysPatt{L}{\Gamma}{\Core{C}{p_1 ... p_n}}{A}
  }

  \Inference[p-sugar]{
    \SaysRule{L}{m}{t_1,...,t_n}{t} \\
    \SaysPatt{L}{\Gamma}{p_1}{t_1} \;\cdots\; \SaysPatt{L}{\Gamma}{p_n}{t_n}
  }{
    \SaysPatt{L}{\Gamma}{\Surf{m}{p_1 ... p_n}}{t}
  }

  \Inference[p-empty]{}{
    \SaysPatt{L}{\Gamma}{[\epsilon]}[t]
  }

  \Inference[p-cons]{
    \SaysPatt{L}{\Gamma}{p}{t} \\
    \SaysPatt{L}{\Gamma}{[ps]}{[t]}
  }{
    \SaysPatt{L}{\Gamma}{[\Cons{p}{ps}]}{[t]}
  }

  \Inference[p-star]{
    l \mapsto [\Gamma'] \in \Gamma & \SaysPatt{L}{\Gamma'}{p}{t}
  }{
    \SaysPatt{L}{\Gamma}{[\Rep{p}{l}]}{[t]}
  }
\end{multicols}

\fbox{$\SaysRule{L}{m}{t,...,t}{t}$}
\Inference[g-sugar]{
  \DsRuleFancy
      {m}
      {(p_{11},...,p_{1n});\Gamma_1;F_1 \To p_1'}
      {(p_{k1},...,p_{kn});\Gamma_k;F_k \To p_k'}
      \in L \\
  \text{The cases are exhaustive over $t_1,...,t_n$ in $G$} \\
  \SaysPatt{L}{\Gamma_{i}}{p_{ij}}{t_j}
    \text{ for each $i \in 1..k, j \in 1..n$} \\
  \SaysPatt{L}{\Gamma_{i}}{p_i'}{t}
    \text{ for each $i \in 1..k$}
}{
  \SaysRule{L}{m}{t_1,...,t_n}{t}
}

\fbox{$\SaysEnv{L}{\gamma}{\Gamma}$}
\Inference[$\gamma$-env]{
  \SaysExpr{L}{e_1}{t_1} \;\cdots\; \SaysExpr{e_n}{t_n} \\
  \SaysEnv{L}{\gamma_{11}}{\Gamma_1} \;\cdots\; \SaysEnv{L}{\gamma_{1j}}{\Gamma_1} \\
  ... \\
  \SaysEnv{L}{\gamma_{m1}}{\Gamma_m} \;\cdots\; \SaysEnv{L}{\gamma_{mk}}{\Gamma_m} \\
}{
  \SaysEnv{L}{
    \begin{cases}
      \PVarA_1 \mapsto e_1,\,...,\,\PVarA_n \mapsto e_n \\
      i_1 \mapsto [\gamma_{11},\,...,\,\gamma_{1j}] \\
      ... \\
      i_m \mapsto [\gamma_{m1},\,...,\,\gamma_{mk}]
    \end{cases}
  }{
    \Gamma',
    \begin{cases}
      \PVarA_1: t_1,\,...,\,\PVarA_n: t_n, \\
      %\PVarA_1': t_1',\,...,\,\PVarA_{n'}': t_{n'}' \\
      i_1 \mapsto [\Gamma_1],\,...,\,i_m \mapsto [\Gamma_m], \\
      %i_1' \mapsto [\Gamma_1'],\,...,\,i_{m'}' \mapsto [\Gamma_{m'}']
    \end{cases}
  }
}

\caption{Grammar Checking}
\end{figure}

\subsection{Type Soundness}

We prove soundness by way of progress + preservation:
\begin{theorem}[Soundness]
  If $\SaysExpr{L}{e}{t}$, then
  $\SaysSteps{L}{e}{v}$ where $\SaysExpr{L}{v}{t}$, or $e$ runs forever.
\end{theorem}
\begin{proof}
\Cref{thm:progress} (progress) and \cref{thm:preservation}
(preservation) together imply that either:
(i) $e$ is a value, or (ii) $\SaysStep{L}{e}{e'}$ and $\SaysExpr{L}{e'}{t}$.
Apply this repeatedly. Either $e$ eventually steps to a value $v$, and
has remained the same type $t$ throughout the evaluation, or $e$ never
halts.
\end{proof}

\begin{lemma}[Progress] \label{thm:progress}
  If $\SaysExpr{L}{e}{t}$, then
  $\SaysStep{L}{e}{e'}$, or $e$ is a value.
\end{lemma}
\begin{proof}
  First, verify that our evaluation contexts include every case that
  isn't a value. Thus either $e$ is a value and we are done, or $e$
  contains a redex: $e=E[\Surf{m}{e_1 ... e_n}]$.
  In the latter case, we will show that $e$ can take a step because
  the eval-expand rule applies. There are two premises that need to be
  satisfied:
  \begin{itemize}
    \item First, we must show that $m$ is bound in $L$. Since $e$
      type-checked, it must be: the only rule which can type-check a
      sugar invocation is p-sugar; this in turn must use rule
      g-sugar; finally g-sugar requires that $m \in L$.
    \item Second, we must show that the pattern match of $e_1,...,e_n$
      succeeds on any case $(p_1,...,p_n);\Gamma;F$ of the desugaring
      rule. By \cref{thm:exhaustion}, it does.
  \end{itemize}
\end{proof}

\begin{assumption}[Exhaustion] \label{thm:exhaustion}
  If the set of cases in a desugaring rule are exhaustive over
  $t_1,...,t_n$ according to our exhaustion checking algorithm, then
  for every possible argument list $e_1,...,e_n$ that matches the
  given types (i.e., $\SaysExpr{L}{e_1}{t_1},...,\SaysExpr{L}{e_n}{t_n}$),
  there is a case $c_i$ such that $e_1,...,e_n$ successfully matches
  against $c_i$. [TODO: prove]
\end{assumption}
\begin{proof}
  \emph{Not given}. We have not stated our exhaustion checking
  algorithm here, and so cannot prove it correct. We believe it is
  straightforward (if tedious).
\end{proof}

\begin{lemma}[Preservation] \label{thm:preservation}
  If $\SaysExpr{L}{e}{t}$ and $\SaysStep{L}{e}{e'}$, then $\SaysExpr{L}{e'}{t}$.
\end{lemma}
\begin{proof}
  Since $e$ can take an expansion step, it must have a redex (via
  eval-ctx): $e = E[\Surf{m}{e_1 ... e_n}]$. And furthermore (by eval-expand) $m$
  must be bound in $L$, and there must be a first case of $m$ that
  matches $e$.  Call it $c_i = (p_1,...,p_n);\Gamma \To p'$. Then:
  \begin{ProofTable}
  By eval-case: & $\SaysMatch{L}{e_i}{p_i}{\gamma_i}$
    for some $\gamma_i$ for each $i$ & (1) \\
  and & $\SaysSubs{F}{\gamma_i \DisjUnion ...
    \DisjUnion \gamma_n}{p'}{e'}$ & (2) \\
  and & $\SaysStep{L}{E[e]}{E[e']}$ \\
  By e-sugar: & $\SaysExpr{L}{\Surf{m}{e_1 ... e_n}}{t}$ \\
  and & $\SaysRule{L}{m}{t_1 ... t_n}{t}$ \\
  and & $\SaysExpr{L}{e_i}{t_i}$ for each $i$ & (3) \\
  By g-sugar: & $\SaysPatt{L}{\Gamma}{p_i}{t_i}$ for each $i$ & (4) \\
  and & $\SaysPatt{L}{\Gamma}{p'}{t}$ & (5)
  \end{ProofTable}
  By \cref{thm:matching} with (1), (3), and (4),
  $\SaysEnv{\gamma_i}{\Gamma}$ for each $i$. By \cref{thm:union},
  $\SaysEnv{\gamma_1 \DisjUnion ... \gamma_n}{\Gamma}$.
  Finally, by \cref{thm:substitution} with that last fact together
  with (2) and (5), $\SaysExpr{L}{e'}{t}$.
\end{proof}

\begin{lemma}[Union of Substitutions] \label{thm:union}
  If $\SaysEnv{L}{\gamma_1}{\Gamma}$ and $\SaysEnv{L}{\gamma_2}{\Gamma}$,
  then $\SaysEnv{L}{\gamma_1 \DisjUnion \gamma_2}{\Gamma}$.
\end{lemma}
\begin{proof}
  [TODO]
\end{proof}

\begin{lemma}[Matching] \label{thm:matching}
  If $\SaysPatt{L}{\Gamma}{p}{t}$
  and $\SaysExpr{L}{e}{t}$
  and $\SaysMatch{F}{e}{p}{\gamma}$,
  then $\SaysEnv{L}{\gamma}{\Gamma}$
\end{lemma}
\begin{proof}
  Induction on $p$.
  \begin{description}
  \item[$p = string$]
    \begin{ProofTable}
      By p-str: & $\SaysPatt{L}{\Gamma}{string}{\TString}$ & fixes $t$ \\
      By m-str: & $\SaysMatch{F}{string}{\TString}{\EmptySubs}$
        & fixes $\gamma$
    \end{ProofTable}
    Finally, by $\gamma$-env, $\SaysEnv{F}{\EmptySubs}{\Gamma}$
    (this applies for any $\Gamma$).
  \item[$p = x \not\in F$] (Analogous.)
  \item[$p = x \in F$] By p-refn or p-decl, 
    $\Gamma = \{\}$ and $t$ is {\TRefn} or {\TDecl}.
    By m-fresh, $e = y$ for some fresh name $y$, and $\gamma = \{\}$.
    And the conclusion follows: $\SaysEnv{L}{\{\}}{\{\}}$. [TODO]
  \item[$p = \PVarA$]
    \begin{ProofTable}
      By p-pvar: & $\SaysPatt{L}{\Gamma}{\PVarA}{t}$ & fixes $t$ \\
      and & $\PVarA: t \in \Gamma$ & (1) \\
      By m-pvar: & $\SaysMatch{F}{e}{\alpha}{\{\PVarA \mapsto e\}}$
        & fixes $\gamma$
    \end{ProofTable}
    Finally, using $\gamma$-env on the premise $\SaysExpr{L}{e}{t}$
    gives that $\SaysEnv{L}{\gamma}{\{\PVarA: t\}},\Gamma'$ for any
    $\Gamma'$. By (1), this is the form of $\Gamma$, so we can set
    $\Gamma'$ such that $\Gamma = {\{\PVarA: t\}},\Gamma'$, and we are done.
  \item[$p = \Core{C}{p_1 ... p_n}$]
    \begin{ProofTable}
      By p-con: & $\SaysPatt{L}{\Gamma}{\Core{C}{p_1 ... p_n}}{A}$ & fixes $t$ \\
      and & $A \mapsto \Core{C}{t_1 ... t_n} \in L$ \\
      and & $\SaysPatt{L}{\Gamma}{p_i}{t_i}$ for each $i$ & (1) \\
      By m-con: &
        $\SaysMatch{F}{\Core{C}{e_1 ... e_n}}{\Core{C}{p_1 ... p_n}}{\gamma}$
        & fixes $e$ \\
      and & $\SaysMatch{F}{e_i}{p_i}{\gamma_i}$ for each $i$ & (2) \\
      and & $\gamma = \gamma_1 \DisjUnion ... \DisjUnion \gamma_n$ \\
      By e-con: & $\SaysExpr{L}{\Core{C}{e_1 ... e_n}}{A}$ \\
      and & $\SaysExpr{L}{e_i}{t_i}$ for each $i$ & (3) \\
    \end{ProofTable}
    Applying the I.H. to (1), (2), and (3) yeilds that
    $\SaysEnv{L}{\gamma_i}{\Gamma}$.
    By \cref{thm:union}, $\SaysEnv{L}{\gamma}{\Gamma}$.
  \item[$p = \Surf{m}{p_1 ... p_n}$] [FILL]
  \item[$p = [\epsilon{]}$] [TODO] By m-empty, $\gamma = \{\}$.
    By p-empty, $\Gamma = \{\}$. The goal follows: $\SaysEnv{L}{\{\}}{\{\}}$.
  \item[$p = [p,ps{]}$] [FILL]
  \item[$p = [\Rep{p}{l'}{]}$]
    \begin{ProofTable}
      By p-star: & $\SaysPatt{L}{\Gamma}{\Rep{p}{l'}}{[t]}$ & fixes $t$ \\
      and & $l' \mapsto [\Gamma'] \in \Gamma$ & (1) \\
      and & $\SaysPatt{L}{\Gamma'}{p}{t}$ & (2) \\
      By m-star: & $\SaysMatch{F}{[e_1 ... e_n]}{[\Rep{p}{l}]}
        {\{l' \mapsto [\gamma_1 \DisjUnion ... \gamma_n]\}}$
        & fixes $e$, $\gamma$ \\
      and & $\SaysMatch{F}{e_i}{p}{\gamma_i}$ for each $i$ & (3) \\
      By e-list: & $\SaysExpr{L}{[e_1 ... e_n]}{[t]}$ \\
      and & $\SaysExpr{L}{e_i}{t_i}$ & (4)
    \end{ProofTable}
    By the I.H. together with (2), (3), and (4),
    $\SaysEnv{L}{\gamma_i}{\Gamma'}$ for each $i$.
    By \cref{thm:union},
    $\SaysEnv{L}{\gamma_1 \DisjUnion ... \gamma_n}{\Gamma'}$.
    Finally, by $\gamma$-env,
    $\SaysEnv{L}{\{l' \mapsto [\gamma_1 \DisjUnion ... \gamma_n]\}}
      {\{l' \mapsto [\Gamma']\}}$, which is compatible with the
      specification of $\Gamma$ in (1).
  \end{description}
\end{proof}

\begin{lemma}[Substitition] \label{thm:substitution}
  If $\SaysEnv{L}{\gamma}{\Gamma}$
  and $\SaysPatt{L}{\Gamma}{p}{t}$,
  and $\SaysSubs{F}{\gamma}{p}{e}$,
  then $\SaysExpr{L}{e}{t}$.
\end{lemma}
\begin{proof}
  Induction on $p$.
  \begin{description}
  \item[$p = string$]
  \item[$p = string$] By s-str, $\SaysSubs{F}{\gamma}{p}{string}$, so $e=string$.
    By p-str, $\SaysPatt{L}{\Gamma}{p}{\TString}$, so $t=\TString$.
    Finally, by e-str, $\SaysExpr{L}{e}{\TString}$ as desired.
  \item[$p = x \not\in F$] (Analogous.)
  \item[$p = x \in F$] By s-fresh, $e = y$ for some fresh name $y$.
    By p-refn or p-decl, $t$ is either {\TRefn} or {\TDecl}.
    Our goal $\SaysExpr{L}{y}{t}$ follows by either e-refn or e-decl,
    respectively.
  \item[$p = \PVarA$] By rule s-pvar, $\PVarA \mapsto e \in \gamma$.
    By $\gamma$-env, $\alpha \mapsto t \in \Gamma$ and $\SaysExpr{L}{e}{t}$.
    Which is our goal; we are done.
    (Note that by $\gamma$-env, $\Gamma$ may have many \emph{other},
    unnecessary, bindings to pattern variables, but it must \emph{at least}
    contain a correct binding for $\alpha$.)
  \item[$p = \Core{C}{p_1 ... p_n}$]
    \begin{ProofTable}
      By p-con: & $\SaysPatt{L}{\Gamma}{\Core{C}{p_1 ... p_n}}{A}$ & fixes $t$ \\
      and & $A \mapsto \Core{C}{t_1 ... t_n} \in L$ & (1) \\
      and & $\SaysPatt{L}{\Gamma}{p_i}{t_i}$ for each $i$ & (2) \\
      By s-con: & $\SaysSubs{F}{\gamma}{\Core{C}{p_1 ... p_n}}{\Core{C}{e_1 ... e_n}}$
        & fixes $e$ \\
      and & $\SaysSubs{F}{\gamma}{p_i}{e_i}$ for each $i$ & (3)
    \end{ProofTable}
    Using the I.H. with (2) and (3) gives that
    $\SaysExpr{L}{e_i}{t_i}$ for each $i$.
    Using e-con on that fact together with (1) gives that
    $\SaysExpr{L}{\Core{C}{e_1 ... e_n}}{A}$, so we are done.
  \item[$p = \Surf{m}{p_1 ... p_n}$]
    \begin{ProofTable}
      By s-sugar: & $\SaysSubs{F}{\gamma}{\Surf{m}{p_1 ... p_n}}{\Surf{m}{e_1 ... e_n}}$
        & fixes $e$ \\
      and & $\SaysSubs{F}{\gamma}{p_i}{e_i}$ for each $i$ & (1) \\
      By p-sugar: & $\SaysPatt{L}{\Gamma}{\Surf{m}{p_1 ... p_n}}{t}$
        & fixes $t$ \\
      and & $\SaysPatt{L}{\Gamma}{p_i}{t_i}$ for each $i$ & (2) \\
      and & $\SaysRule{L}{m}{t_1,...,t_n}{t}$ & (3)
    \end{ProofTable}
    Using the I.H. with (1) and (2) gives that
    $\SaysExpr{L}{e_i}{t_i}$ for each $i$.
    Finally, using e-sugar on that fact together with (3) gives that
    $\SaysExpr{L}{\Surf{m}{e_1 ... e_n}}{t}$.
  \item[$p = [\epsilon{]}$]
    By s-empty, $\SaysSubs{F}{\gamma}{p}{[]}$, so $e=[]$.
    By p-empty, $\SaysPatt{L}{\EmptyEnv}{[\epsilon]}{[t]}$ (for some $t$).
    Finally, by e-list, $\SaysExpr{L}{[]}{[t]}$.
  \item[$p = [p,ps{]}$]
    \begin{ProofTable}
      By s-cons: & $\SaysSubs{F}{\gamma}{p}{e_1}$ & (1) \\
      and & $\SaysSubs{F}{\gamma}{[ps]}{[e_2 ... e_n]}$ & (2) \\
      and & $\SaysSubs{F}{\gamma}{[p, ps]}{[e_1 e_2 ... e_n]}$ & fixes $e$ \\
      By p-cons: & $\SaysPatt{L}{\Gamma}{p}{t}$ & (3) \\
      and & $\SaysPatt{L}{\Gamma}{[ps]}{[t]}$ & (4) \\
      and & $\SaysPatt{L}{\Gamma}{[p, ps]}{[t]}$ & fixes $t$
    \end{ProofTable}
    We can apply the I.H. using (1) and (3) and the assumption
    $\SaysEnv{L}{\gamma}{\Gamma}$ to get that $\SaysExpr{L}{e_1}{t}$.
    Likewise, the I.H. with (2) and (4) gives
    $\SaysExpr{L}{[e_2 ... e_n]}{[t]}$.
    By e-list (in reverse), $\SaysExpr{L}{e_2}{t} \cdots \SaysExpr{L}{e_n}{t}$.
    Finally, by e-list (forward), $\SaysExpr{L}{[e_1 e_2 ... e_n]}{[t]}$.
  \item[$p = [\Rep{p}{l}{]}$]
    \begin{ProofTable}
      By s-star: & $\SaysSubs{F}{\gamma}{[\Rep{p}{l}]}{[e_1 ... e_n]}$ & fixes $e$ \\
      and & $l \mapsto [\gamma_1 ... \gamma_n] \in \gamma$ \\
      and & $\SaysSubs{F}{\gamma_i}{p}{e_i}$ for each $i$ & (1) \\
      By $\gamma$-env: & $l \mapsto [\Gamma'] \in \Gamma$ \\
      and & $\SaysEnv{L}{\gamma_i}{\Gamma'}$ for each $i$ & (2) \\
      By p-star: & $\SaysPatt{L}{\Gamma}{[\Rep{p}{l}]}{[t]}$ & fixes $t$ \\
      and & $\SaysPatt{L}{\Gamma'}{p}{t}$ & (3)
    \end{ProofTable}
    Using the I.H. with (1), (2), and (3) proves that
    $\SaysExpr{L}{e_i}{t}$.
    Then, by e-list, $\SaysExpr{L}{[e_1 ... e_n]}{[t]}$ as desired.
  \end{description}
\end{proof}

\section{Scope Checking}

(See \cref{fig:scope}.)

\begin{figure}
\fbox{$\SaysScope{\Sigma}{e}
  {\{\Decl{x}\}}
  {\{\Refn{x}\}}
  {\{\Decl{x}\}}
  {\{\Refn{x}\mapsto\Decl{x}\}}$}

\Inference[scope-e-decl]{}{
  \SaysScope{\Sigma}{\Decl{x}}{\{\Decl{x}\}}{\{\}}{\{\Decl{x}\}}{\{\}}
}

\Inference[scope-e-refn]{}{
  \SaysScope{\Sigma}{\Refn{x}}{\{\}}{\{\Refn{x}\}}{\{\}}{\{\}}
}

\Inference[scope-e-con]{
  \Sigma[C] = \sigma \vspace{0.1em} \IS\\
  \SaysScope{\Sigma}{e_i}{P_i}{R_i}{B_i} \text{ for } i \in 1..n \IS\\
  S = \SetSuchThat{
    \Decl[a]{x} \mapsto \Decl[b]{x}
  }{
    \Decl[a]{x} \in P_i,\;
    \Decl[b]{x} \in P_j,\;
    \Bind{\sigma}{i}{j}
  } \IS\\
  B = \SetSuchThat{
    \Refn[a]{x} \mapsto \Decl[b]{x}
  }{
    \Refn[a]{x} \in R_i,\;
    \Decl[b]{x} \in P_j,\;
    \Bind{\sigma}{i}{j},\;
    \Decl[b]{x} \not\in \Domain{S}
  } \IS\\
  R = \SetSuchThat{
    \Refn[a]{x}
  }{
    \Refn[a]{x} \in R_i,\;
    \NotExists{\Decl[b]{x}}{\Refn[a]{x} \mapsto \Decl[b]{x} \in B\}}
  } \IS\\
  P = \SetSuchThat{
    \Decl[a]{x}
  }{
    \Decl[a]{x} \in P_i,\;
    \Prov{\sigma}{i},\;
    \Decl[a]{x} \not\in \Domain{S}
  }
}{
  \SaysScope{\Sigma}{\Core{C}{e_1 ... e_n}}{P}{R}{B}
}


\Inference[scope-p-con]{
  \Sigma[C] = \sigma \vspace{0.1em} \IS\\
  \SaysScope{\Sigma}{p_i}{P_i}{R_i}{B_i} \text{ for } i \in 1..n \IS\\
  S = \SetSuchThat{
    a \mapsto b
  }{
    a \in P_i,\;
    b \in P_j,\;
    \Bind{\sigma}{i}{j},\;
    \SaysCanShadow{F}{a}{b}
  } \IS\\
  B = \SetSuchThat{
    a \mapsto b
  }{
    a \in R_i,\;
    b \in P_j,\;
    \Bind{\sigma}{i}{j},\;
    b \not\in \Domain{S},\;
    \SaysCanBind{F}{a}{b}
  } \IS\\
  R = \SetSuchThat{
    a
  }{
    a \in R_i,\;
    (\NotExists{b}{a \mapsto b \in B\}}
    \text{ or $a$ is a pattern var})
  } \IS\\
  P = \SetSuchThat{
    a
  }{
    a \in P_i,\;
    \Prov{\sigma}{i},\;
    a \not\in \Domain{S}
  }
}{
  \SaysScope{\Sigma}{\Core{C}{p_1 ... p_n}}{P}{R}{B}
}

Two checks to make: fresh vars don't bind to non-fresh vars, and named
vars only bind to vars of the same name:
\begin{multicols}{3}
  \Inference{}{
    \SaysCanBind{F}{\Refn[1]{x}}{\Decl[2]{x}}
  }
  \Inference{
    x \not\in F
  }{
    \SaysCanBind{F}{\Refn[1]{x}}{\PVarA}
  }
  \Inference{
    x \not\in F
  }{
    \SaysCanBind{F}{\PVarA}{\Decl[1]{x}}
  }
  \Inference{}{
    \SaysCanBind{F}{\PVarA}{\PVarB}
  }
  \Inference{}{
    \SaysCanShadow{F}{\Decl[1]{x}}{\Decl[2]{x}}
  }
  \Inference{
    x \not\in F
  }{
    \SaysCanShadow{F}{\Decl[1]{x}}{\PVarA}
  }
\end{multicols}

\caption{Scope Checking Rules}
\label{fig:scope}
\end{figure}







\chapter{Appendix}

\subsection{Terms}

We will call \textsc{ast} terms \emph{expressions} and write them in
s-expression form. Atomic terms are either variables or literals
(i.e. syntactic constants), and compound terms are built with
\emph{term constructors} $P$:

\begin{jtable}
  $e$
  &$::=$& $\lit{lit}$ &(literal) \\
  &$|$&   $\var{x}$ &(variable) \\
  &$|$&   $\expr{P}{e_1 ... e_n}$ &(\textsc{ast} node)
\end{jtable}

\subsection{Tree Grammars}

A \emph{tree grammar} [CITE] is to trees as a context-free grammar is
to strings. Thus it can be viewed either as a set of instructions for
how to iteratively and nondeterministically rewrite a starting
\emph{nonterminal} to a final tree; \emph{or} it can be viewed as a
specification of a grammar that a tree may or may not follow. We will
take the latter view.

Definitionally, a tree grammar consists of a number of
\emph{productions} that map \emph{nonterminals} $s$ to
\name{patterns}:
\begin{jtable}
  $G$
  &$::=$& $\left\{ \begin{array}{l}
    \production{s_1}{\name{pattern}_1} \\
    \ddd \\
    \production{s_n}{\name{pattern}_n}
    \end{array}\right.$ \\
  \\
  $\name{pattern}$
  &$::=$& $\expr{P}{s_1 \dd s_n}$ &(regular pattern) \\
  &$|$&
  $\exprs{P}{s_1 \dd s_n}{s_{n+1}}$
  &(ellipses pattern) \\
  &$|$&   $\constName{literal}$ &(matches literals) \\
  &$|$&   $\constName{var}$ &(matches variables)
\end{jtable}

The \emph{meaning} of a tree grammar (again, we are viewing the
grammar as a \emph{specification}) is that for each production
``$\production{s}{\name{pattern}}$'', if a term matches the
\name{pattern}, then it also matches the nonterminal $s$. Formally:

\[
\fbox{$\saysG{G}{e}{s}$}
\]

\[
\inference
    [G-literal]
    {}
    {\saysG{G}{\lit{lit}}{\constName{literal}}}
\quad
\inference
    [G-variable]
    {}
    {\saysG{G}{\var{x}}{\constName{variable}}}
\]
    
\[
\inference
    [G-node]
    {\Forall{i \in 1..n} \saysG{G}{e_i}{s_i} \\
      \production{s}{\expr{P}{s_1 \dd s_n}} \in G}
    {\saysG{G}{\expr{P}{e_1 \dd e_n}}{s}}
\quad
\inference
    [G-ellipses]
    {\Forall{i \in 1..n} \saysG{G}{e_i}{s_i} \\
      \Forall{j \in 1..k} \saysG{G}{e_{n+j}}{s} \\
      \production{s}{\exprs{P}{s_1 \dd s_n}{s}} \in G}
    {\saysG{G}{\expr{P}{e_1 \dd e_n \dd e_{n+k}}}{s}}
\]

\end{document}
