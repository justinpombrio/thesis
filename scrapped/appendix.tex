
\chapter{Appendix}


% Expressions & Grammars
\newcommand{\lit}[1]{\textbf{#1}}
\newcommand{\expr}[2]{(#1\,#2)}
\newcommand{\var}[1]{\textrm{\textsc{#1}}}

% Grammars
\newcommand{\exprs}[3]{(#1\,#2\,#3^{*})}
\newcommand{\production}[2]{#1 \leftarrow #2}
\newcommand{\saysG}[3]{#1 \vdash #2\,:\,#3}

% Scope
\newcommand{\SaysScopeCheck}[6]{#1 \vdash #2 : #3 ; #4 ; #5 ; #6}


\subsection{Terms}

We will call \textsc{ast} terms \emph{expressions} and write them in
s-expression form. Atomic terms are either variables or literals
(i.e. syntactic constants), and compound terms are built with
\emph{term constructors} $P$:

\begin{Table}
  $e$
  &$::=$& $\lit{lit}$ &(literal) \\
  &$|$&   $\var{x}$ &(variable) \\
  &$|$&   $\expr{P}{e_1 ... e_n}$ &(\textsc{ast} node)
\end{Table}

\subsection{Tree Grammars}

A \emph{tree grammar} [CITE] is to trees as a context-free grammar is
to strings. Thus it can be viewed either as a set of instructions for
how to iteratively and nondeterministically rewrite a starting
\emph{nonterminal} to a final tree; \emph{or} it can be viewed as a
specification of a grammar that a tree may or may not follow. We will
take the latter view.

Definitionally, a tree grammar consists of a number of
\emph{productions} that map \emph{nonterminals} $s$ to
\emph{patterns}:
\begin{Table}
  $G$
  &$::=$& $\left\{ \begin{array}{l}
    \production{s_1}{\emph{pattern}_1} \\
    \ddd \\
    \production{s_n}{\emph{pattern}_n}
    \end{array}\right.$ \\
  \\
  $\emph{pattern}$
  &$::=$& $\expr{P}{s_1 \dd s_n}$ &(regular pattern) \\
  &$|$&
  $\exprs{P}{s_1 \dd s_n}{s_{n+1}}$
  &(ellipses pattern) \\
  &$|$&   $\emph{literal}$ &(matches literals) \\
  &$|$&   $\emph{var}$ &(matches variables)
\end{Table}

The \emph{meaning} of a tree grammar (again, we are viewing the
grammar as a \emph{specification}) is that for each production
``$\production{s}{\emph{pattern}}$'', if a term matches the
\emph{pattern}, then it also matches the nonterminal $s$. Formally:

\[
\fbox{$\saysG{G}{e}{s}$}
\]

\[
\inference
    [G-literal]
    {}
    {\saysG{G}{\lit{lit}}{\emph{literal}}}
\quad
\inference
    [G-variable]
    {}
    {\saysG{G}{\var{x}}{\emph{variable}}}
\]
    
\[
\inference
    [G-node]
    {\Forall{i \in 1..n} \saysG{G}{e_i}{s_i} \\
      \production{s}{\expr{P}{s_1 \dd s_n}} \in G}
    {\saysG{G}{\expr{P}{e_1 \dd e_n}}{s}}
\quad
\inference
    [G-ellipses]
    {\Forall{i \in 1..n} \saysG{G}{e_i}{s_i} \\
      \Forall{j \in 1..k} \saysG{G}{e_{n+j}}{s} \\
      \production{s}{\exprs{P}{s_1 \dd s_n}{s}} \in G}
    {\saysG{G}{\expr{P}{e_1 \dd e_n \dd e_{n+k}}}{s}}
\]


\section{Scope Checking}

(See \cref{fig:scope}.)

\begin{figure}
\TypeLabel{\SaysScopeCheck{\Sigma}{e}
  {\{\Decl{x}\}}
  {\{\Refn{x}\}}
  {\{\Decl{x}\}}
  {\{\Refn{x}\mapsto\Decl{x}\}}}

\Inference[scope-e-decl]{}{
  \SaysScopeCheck{\Sigma}{\Decl{x}}{\{\Decl{x}\}}{\{\}}{\{\Decl{x}\}}{\{\}}
}

\Inference[scope-e-refn]{}{
  \SaysScopeCheck{\Sigma}{\Refn{x}}{\{\}}{\{\Refn{x}\}}{\{\}}{\{\}}
}

\Inference[scope-e-con]{
  \Sigma[C] = \sigma \vspace{0.1em} \IS\\
  \SaysScopeCheck{\Sigma}{e_i}{P_i}{R_i}{B_i} \text{ for } i \in 1..n \IS\\
  S = \SetSuchThat{
    \Decl[a]{x} \mapsto \Decl[b]{x}
  }{
    \Decl[a]{x} \in P_i,\;
    \Decl[b]{x} \in P_j,\;
    \Bind{\sigma}{i}{j}
  } \IS\\
  B = \SetSuchThat{
    \Refn[a]{x} \mapsto \Decl[b]{x}
  }{
    \Refn[a]{x} \in R_i,\;
    \Decl[b]{x} \in P_j,\;
    \Bind{\sigma}{i}{j},\;
    \Decl[b]{x} \not\in \Domain{S}
  } \IS\\
  R = \SetSuchThat{
    \Refn[a]{x}
  }{
    \Refn[a]{x} \in R_i,\;
    \NotExists{\Decl[b]{x}}{\Refn[a]{x} \mapsto \Decl[b]{x} \in B\}}
  } \IS\\
  P = \SetSuchThat{
    \Decl[a]{x}
  }{
    \Decl[a]{x} \in P_i,\;
    \Prov{\sigma}{i},\;
    \Decl[a]{x} \not\in \Domain{S}
  }
}{
  \SaysScopeCheck{\Sigma}{\Core{C}{e_1 ... e_n}}{P}{R}{B \cup B_1 \cup \ddd \cup B_n}
}


\Inference[scope-p-con]{
  \Sigma[C] = \sigma \vspace{0.1em} \IS\\
  \SaysScopeCheck{\Sigma}{p_i}{P_i}{R_i}{B_i} \text{ for } i \in 1..n \IS\\
  S = \SetSuchThat{
    a \mapsto b
  }{
    a \in P_i,\;
    b \in P_j,\;
    \Bind{\sigma}{i}{j},\;
    \SaysCanShadow{F}{a}{b}
  } \IS\\
  B = \SetSuchThat{
    a \mapsto b
  }{
    a \in R_i,\;
    b \in P_j,\;
    \Bind{\sigma}{i}{j},\;
    b \not\in \Domain{S},\;
    \SaysCanBind{F}{a}{b}
  } \IS\\
  R = \SetSuchThat{
    a
  }{
    a \in R_i,\;
    (\NotExists{b}{a \mapsto b \in B\}}
    \text{ or $a$ is a pattern var})
  } \IS\\
  P = \SetSuchThat{
    a
  }{
    a \in P_i,\;
    \Prov{\sigma}{i},\;
    a \not\in \Domain{S}
  }
}{
  \SaysScopeCheck{\Sigma}{\Core{C}{p_1 ... p_n}}{P}{R}{B \cup B_1 \cup \ddd \cup B_n}
}

Two checks to make: fresh vars don't bind to non-fresh vars, and named
vars only bind to vars of the same name:
\begin{multicols}{3}
  \Inference{}{
    \SaysCanBind{F}{\Refn[1]{x}}{\Decl[2]{x}}
  }
  \Inference{
    x \not\in F
  }{
    \SaysCanBind{F}{\Refn[1]{x}}{\PVarA}
  }
  \Inference{
    x \not\in F
  }{
    \SaysCanBind{F}{\PVarA}{\Decl[1]{x}}
  }
  \Inference{}{
    \SaysCanBind{F}{\PVarA}{\PVarB}
  }
  \Inference{}{
    \SaysCanShadow{F}{\Decl[1]{x}}{\Decl[2]{x}}
  }
  \Inference{
    x \not\in F
  }{
    \SaysCanShadow{F}{\Decl[1]{x}}{\PVarA}
  }
\end{multicols}

\caption{Scope Checking Rules}
\label{fig:scope}
\end{figure}
